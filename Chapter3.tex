\documentclass[main.tex]{subfiles}
\begin{document}

\section{Introduction}

The Cherenkov Telescope Array(CTA)\cite{CTA} is the next-generation ground-based observatory for $\gamma$-ray astronomy. It is the result of the combination of efforts from all the $\gamma$-ray community, to push the field of \gls{vhe} astrophysics to new limits. It will consist on two observatories of \glspl{iact} located in the Northern hemisphere, in the Roque de los Muchachos Observatory in the island of La Palma (Canary Islands, Spain) and in the Southern hemisphere in the Paranal desert (Chile). With this configuration, \gls{cta} will be the first ground-based observatory to be able to survey the full sky. \gls{cta} will have five to ten times more sensitivity than current instruments in the wider energy range ever reached, from 20 GeV to 300 TeV. Its fine angular resolution up to 2 arcseconds, with a field of view of $\sim 10º$ will allow the observation of fine structures in extended $\gamma$-ray sources. Its energy resolution with an uncertainty less than 10\% will make \gls{cta} capable of study features in the sources spectra, such as lines and cutoffs.\\
To reach such improvements with respect to the current generation of \glspl{iact} observatories, \gls{cta} will count with more than 100 telescopes in three sizes, each designed to improve the performance in a specific spectral range. \glspl{lst} have the bigger size, with a 32 m diameter mirror. They focus on the lowest part of the \gls{vhe} spectrum ($\sim$20 GeV to 3 TeV) where $\gamma$-ray showers are common, so a small number of telescopes is enough, but the amount of Cherenkov light is low so big mirrors are required to collect the maximum number of photons. \gls{mst} have an intermediate size, with a diameter of$\sim 10m$. They have the best performance at the range between $\sim 100 GeV$ and 50 TeV. \gls{sst} have much smaller mirrors ($\sim$ 4 m), but many of them will be installed to get big collection areas and be able to capture the highest energy photons (up to 300 TeV), which are very rare but produce huge amounts of Cherenkov light.\\
Also, thanks to the large number of telescopes, \gls{cta} will allow great flefibility of operations, with different observation modes: with full array for best sensitivity; using subarrays to observe several sources at the same time; and divergent pointing mode where the field of view can be extended up to 20 deg.\\
In this chapter the main features of \gls{cta} are covered. A description of performance is given in section \ref{sec:ctaperformance}. In section \ref{sec:ctatelescopes} the three types of telescopes are described in detail. Section \ref{sec:ctaanalysis} is dedicated to \gls{cta} analysis techniques and tools at different levels.

\section{CTA Requirements and Performance} \label{sec:ctaperformance}

The current generation of Cherenkov telescope observatories consist on arrays of 2 to 5 telescopes which reach sensitivities of about 1\% of the Crab at the range of 0.1-1 TeV. The typical angular resolution is about 0.1º, but it can improve largely for intense point sources (20-30 arc seconds). The goal of \gls{cta} design is to advance the state of the art in $\gamma$-ray astronomy, improving sensitivity, energy range, angular and timing resolution and offering full sky coverage. The performance goals of \gls{cta} are shown in table \ref{tab:CTAgoals}. As a result of these goals, \gls{cta} will account for two unique features: the ability to produce a deep survey of the \gls{vhe} sky, discovering hundreds of new sources; and the possibility to observe the shortest time-scale phenomena, such as flares and jets from black holes in the center of active galaxies or periodic emission like such of pulsars. 
To achieve these improvements, different types of telescopes are required and they should be spreaded over large areas to reach a collection area of several km$^2$ \cite{CTAconcept}.

\begin{table}
  \centering
  \begin{tabular}{ccc}
    \hline
    Diff. sensitivity & at 50 GeV & $8\times10^{-12}$\\
    (erg cm$^2$ s$^{-1}$) & at 1 TeV & $2\times10^{-13}$\\
     & at 50 TeV & $3\times10^{-13}$ (S) / $3\times10^{-12}$ (N)\\\\
    Collection area (m$^{2}$) & at 1 TeV & > $10^4$ \\
    & at 10 TeV & > $10^6$ (S)/ >$5\times10^5$ (N) \\\\
    Angular resolution & at 0.1 TeV & 0.1º \\
    & > 1 TeV & 0.05º \\\\
    Energy resolution & at 50 GeV & $\le 25\%$ \\
    & > 1 TeV & $\le 10\%$ \\\\
    Field of view & at 0.1 TeV & 5º\\
    & at 1 TeV & 8º\\
    & > 10 TeV & 10º\\\\
    Sensitivity in FoV & at 1 TeV flat out to & > 2.5º \\
    & at 1 TeV & 5'' per axis \\\\
    Repointing time & <0.1 TeV  & 20s (goal), 50s(max) \\
    & 0.1-10 TeV & 60s (goal), 90s (max) \\
    \hline
  \end{tabular}
  \caption{Performance goals for CTA observatories, adapted from \cite{CTAconcept}. Sensitivity given in 5 bins per energy. (N)= Northern site, (S) = Southern site.}
  \label{tab:CTAgoals}
\end{table}

The final performance of \gls{cta} will depend on the design of the telescopes, which will be discussed more deeply in next sections, and also in the arrangement of them along the two sites.
To have an estimate of the performance of the observatory and to decide which will be the best configuration (number of telescope of each type and how are them) detailed Monte Carlo simulations were made \cite{2013CTAMonteCarlo}, comprising simulations of shower development in the atmosphere, detector response and event reconstruction.
A deep study of these simulations \cite{2017CTAMCPerformance} for different configurations of telscopes and sites was made to came up with the final array layout illustrated in picture \ref{fig:arraylayout}.

\begin{figure}
\centering
 \includegraphics[width=1\textwidth]{Pictures/Array-Layouts.pdf}
  \caption{Array layout selected for \gls{cta} from \cite{CTAPerformance}. Left is Northern site where also \gls{magic} telescopes are located. Right is Southern location.}
    \label{fig:arraylayout}
\end{figure}

The baseline design for Southern observatory forsees 4 \glspl{lst}, \glspl{mst} and 70 \glspl{sst}. It will be dedicated to the study of galactic sources which emit the most energetic $\gamma$-rays to reach the Earth. Because of the extremely low fluxes, it will cover very a large area ($\sim4$ km$^2$), achieved thanks to the large number of \glspl{sst}, to capture as many events as possible.\\
The Northern observatory will mainly observe extragalctic sources and transient events from which only the less energetic photons reach the Earth due to absorption (see section \ref{sec:absorption}). The final configuration will consist on 4 \glspl{lst} and 15 \glspl{mst}.

The performance of \gls{cta} has been evaluated in terms of the best results for the observation of a point source located at the centre of the field of view of telescopes cameras. A set of metrics are taken into account: Effective collection area, residual cosmic-ray background rate, angular resolution and differential sensitivity. Usually they are calculated after applying cuts in \textit{gammaness} and $\theta^2$ optimized to maximize sensitivity for a set of typical observation times. Gammaness is referred to the gamma-hadron separation of shower events, events with higher gammaness are more likely to be produced by a $\gamma$-ray instead of a hadron. The angle $\theta^2$ is the square of the angle between the reconstructed source position and the true source position.
An overview of the mentioned metrics for performance evaluation is the following:\\

\begin{itemize}
\item \textbf{Effective collection area:} The effective collection area describes the area in which \gls{cta} is able to detect a shower from the point source. The effective collection areas for the assumed point sources are shown in figure \ref{fig:effarea}.\\

\begin{figure}[!htb]
\minipage{0.5\textwidth}
\includegraphics[width=\linewidth]{Pictures/CTA-Performance-prod3b-v2-South-20deg-EffectiveArea.pdf}
\endminipage\hfill
\minipage{0.5\textwidth}
\includegraphics[width=\linewidth]{Pictures/CTA-Performance-prod3b-v2-North-20deg-EffectiveArea.pdf}
\endminipage\hfill
\caption{\label{fig:effarea}Effective collection area for point-like sources for CTA South(left) and CTA North (right) \cite{CTAPerformance}.}
\end{figure}

\item \textbf{Residual cosmic-ray background rate:} The ability of \gls{cta} to reject background events, meaning showers produced by \glspl{cr} instead of $\gamma$s is measured through the residual cosmic-ray background rate. The 99.9\% of the showers reaching the telescopes are produced by \glspl{cr}, so a good background rejection is a key requirement for \glspl{iact}. In figure \ref{fig:bkgrate} the (post-analysis) residual cosmic-ray background rate per square degree vs reconstructed $\gamma$-ray energy is shown. The rate is integrated in each bin, where five bins per decade has been taken. \gls{cta} have such strong background rejection capabilities that the majority of the background in the range 0.2-1.5 TeV is due cosmic-ray electrons and positrons \cite{2017ICRCCTAPerformance} which produce showers much more difficult to differentiate from electromagnetic showers than those produced by hadrons (see section \ref{sec:electroshowers}).\\

  \begin{figure}[!htb]
    \minipage{0.5\textwidth}
    \includegraphics[width=\linewidth]{Pictures/CTA-Performance-prod3b-v2-South-20deg-BackgroundRateSquDeg.pdf}
    \endminipage\hfill
    \minipage{0.5\textwidth}
    \includegraphics[width=\linewidth]{Pictures/CTA-Performance-prod3b-v2-North-20deg-BackgroundRateSquDeg.pdf}
    \endminipage\hfill
    \caption{\label{fig:bkgrate} Residual cosmic-ray background rate vs reconsctructed enregy for CTA South(left) and CTA North (right) \cite{CTAPerformance}.}
  \end{figure}
  
\item \textbf{Angular resolution:} The angular resolution is defined as the angle within which 68\% of reconstructed $\gamma$-rays fall, relative to their true direction. Figure \ref{fig:angres} show the angular resolution vs. reconstructed energy optimized for best point-source sensitivity. It is possible to improve angular resolution at expenses of collection area for a better study of morphological features of bright sources.\\
  
  \begin{figure}[!htb]
    \minipage{0.5\textwidth}
    \includegraphics[width=\linewidth]{Pictures/CTA-Performance-prod3b-v2-Comparison-AngularResolution-OtherInstruments.pdf}
    \endminipage\hfill
    \minipage{0.5\textwidth}
    \includegraphics[width=\linewidth]{Pictures/CTA-Performance-prod3b-v2-North-20deg-AngularResolution.pdf}
    \endminipage\hfill
    \caption{\label{fig:angres} Angular resolution for CTA South compared to other experiments (left), and for CTA North (right) \cite{CTAPerformance}.}
  \end{figure}

\item \textbf{Energy resolution:} The energy resolution $\Delta E/E$ is obtained from the distribution $E_{rec}-E_{true}/E_{true}$, where $E_{true}$ and $E_{rec}$ are the true and reconstructed energies of $\gamma$-ray events. It is defined as the half-width of the interval around 0 which contains the 68\% of the distribution. The energy resolution of \gls{cta} as a function of reconstructed energy is shown in figure \ref{fig:energyres}. Note that for the range of 1-10 TeV the energy resolution is well below the required 10\%.\\
    
  \begin{figure}[!htb]
    \minipage{0.5\textwidth}
    \includegraphics[width=\linewidth]{Pictures/CTA-Performance-prod3b-v2-South-20deg-EnergyResolution.pdf}
    \endminipage\hfill
    \minipage{0.5\textwidth}
    \includegraphics[width=\linewidth]{Pictures/CTA-Performance-prod3b-v2-North-20deg-EnergyResolution.pdf}
    \endminipage\hfill
    \caption{\label{fig:energyres} Energy resolution vs reconstructed energy for CTA South (left) and CTA North (right) \cite{CTAPerformance}.}
  \end{figure}
  
\item \textbf{Differential sensitivity:} The most important feature at evaluating the performance is the differential sensitivity, which indicates the minimum flux needed by \gls{cta} to detect a point-like source at $5\sigma$. The sensitivity is calculated in five energy bins per decade, where it is required that at least 10 $\gamma$-rays are detected and the signal to background ratio is al least 1/20. The cuts in gammaness and source direction ($\theta^2$) are optimized to improve this quantity. The differential sensitivity of \gls{cta} compared to that of other instruments for 50h observation time was shown in figure \ref{fig:ctaperformance}.\\
  
\end{itemize}

\section{CTA Telescopes} \label{sec:ctatelescopes}

While all telescopes share a basic design concept, each of them have particlar features to accomplish the performance requirements for \gls{cta}. In general, subsystems of \glspl{iact} are divided in \textbf{structure}, \textbf{optics} (mirros) and \textbf{camera}. 
In this section an overview of these subsystems and the particularities of each of the three types of \gls{cta} telescopes is given. 

\subsection{Large Size Telescope (LST)}

The purpose of the \gls{lst} is to enhance the sensitivity of \gls{cta} at low energies, reaching an effective threshold down to 20-30 GeV. This energy range has a particular interest because it covers the barely explored gap between the highest energies detcted by satellite experiments, and the lowest by \glspl{iact}. The main science goals for \glspl{lst} is to observe high redshift \glspl{agn}, \glspl{grb}, pulsars and galactic transients. The first prototype for the \glspl{lst} (\gls{lst}1) has been installed in the CTA North site, in La Palma island and is currently being commissioned.\\
A detailed description of \gls{lst} subsystems can be found in \cite{2013LST}, next sections offer a summary of the most remarckable features of the telescope.\\

\subsubsection{\gls{lst} Structure}

The structure of the telescope is divided in several parts: The azimuth system and substructure, the mirror support dish, the elevation system and, camera support structure and the drive system. All parts of the structure are specially designed to make the telescope lightweight and ensure fast repositioning in the case of an alert of a transient event (mainly \glspl{grb}).\\
The azimuth system allows the telescope to turn around its vertital axis. It consist in a circular rail of 24m diameter where six boogies support the structure of the telescope. The azimuth structure of the telescope is a space framed structure made of \gls{cfrp} tubes. Its design is very similar to that of \gls{magic} telescopes, but much lighter. The mirror support dish is a double layer space frame made of \gls{cfrp} tubes arranged in a tetrahedral structure. The diameter of the dish is 23 m and the focal length is 28 m. The elevation system consist on a structure made of heavier steel tubes, located in the backside of the dish, to help compensate the weight of the telescope structure. The camera support structure consists on an arch with three curve sections on each arm, made of \gls{cfrp}. A total of 26 ropes shiften the structure. The arch holds the camera frame, where the camera and the square lids are fixed. The drive system, designed to ensure a fast and precise repositioning, consist on 4 synchronous motors capable to supply a total mechanical power of 190 Kw to move the structure in azimuthal axis and elevation. The different parts of the structure are shown in figure \ref{fig:LST}.
    
  \begin{figure}[!htb]
    \minipage{0.5\textwidth}
    \includegraphics[width=\linewidth]{Pictures/LSTstructure.pdf}
    \endminipage\hfill
    \minipage{0.5\textwidth}
    \includegraphics[width=\linewidth]{Pictures/LST1.pdf}
    \endminipage\hfill
    \caption{\label{fig:LST} Left: Simplified diagram of the structures of \gls{lst} without bogies, mirrors and camera. Azimuth substructure is blue color, the elevation system and the camera support structure are shown in green color and the mirror dish is red color, from \cite{2013LST}. Right: Photo of the first LST installed in CTA North site (Roque de los Muchachos Observatory, La Palma, Spain).}
  \end{figure}
  
  \subsubsection{\gls{lst} Optics}

  The mirror structure is a parabolic shape of 28 m diameter, composed of 198 hexagona mirrors of spherical shape. The structure is made in a way that it guarantees the isochronousity of the optics. The mirrors are manufactured using the cold slump technique where a sandwich structure is composed a soda-lime glass sheet, an aluminum honeycomb box and another glass sheet\cite{2017LST}.
Mirrors are fixed to the telescope structure by three knots, two of them mounted on actuators which allow to adjust each mirro panel in two directions. The \textit{active mirror control} system controls these actuaros in order to vary focal distance and focusing, to account for the possible deformations and bending of the telescope structure. Each mirror segment account for an ifnrarred laser, and two more lasers are mounted in the center of the disk, which constantly point to the left and right of the camera. The directional offset of the mirror facets can be estimated by taking pictures of the spots on a target in fornt of the camera with a high resolution IR CCD camera located in the center of the dish. The achievable positioning resolution can reach $< 5 \mum$ \cite{2013LST}.
  
  \subsubsection{\gls{lst} Camera}

  The camera focal plane instrumentation consist on 256 modules each of one with seven \glspl{pmt}, making a total of 1855 pixels. The \glspl{pmt} are mounted with a 50mm spacing, so in order to collect all the light arriving to the camera, they are surrounded by hexagonal light guides made of a very reflective material (3M ESR). The \gls{pmt} model is Hamamatsu R11920-100-20, specially developed to reach high quantum efficiency (42\%) and low after pulse rates, which go below 0.02 \% above 4 photoelectrons. Each module with seven photodetectors has one readout system based on the \gls{dsr4}. The signal sampled at \mathcal{O}(GHz) is divided in three channels: high gain and low gain channels which are connected to the \gls{dsr4} chips and the trigger channel \cite{2017LST}. The two level trigger electronics (L0 and L1) are mounted on a mezzanine. A slow control board montors the \glspl{pmt} and analogue backplanes are dedicated to trigger and clock propagation.  
  The mechanical structure of the camera consist on several parts shown in figure \ref{fig:LSTcammech}. The modules of seven \gls{pmt} are coupled in ensembles named clusters, which are inserted in the load bearing structure. Slits in the back of the load bearing structure allow to connect the cluster readout electronics to the  backplanes. The design of such modular structures guarantee an easy access to individual modules in case any repair is necessary, allowing to disconect and extract one module while the rest of the camera is functioning.
  The load bearing structure is mounted inside a tubular structure which holds the camera external walls, doors and interfaces with the camera frame. In the front part of the camera there is the front window which can be totally covered by a shutter to avoid the entrance of sunlight during the day which could damage \glspl{pmt}. In the back part of the camera lies all the cabling and auxiliary electronics. The total dimensions of the camera is $2.9\times2.9\times 1$m$^3$ and the weight is less than 2000 kg. Since all the electronics are stored inside the camera, all the structure was designed to accomplish a series of requirements related to the exposure to environental conditions such as the protection to the entrance of dust, water and sunlight and guarantee a stable temperature between 20ºC to 40ºC \cite{2013LSTCamMech}. Temperature regulation is possible thanks to a \textit{cooling system} consisting on a water cooling system based on cold plates and a temperature controlled air flow cooling system.\\
 An \textit{access tower} allow the access to the camera during day time operations when the telescope rests in park position, and also acts as an anchor in case of strong winds. 

\begin{figure}
\centering
 \includegraphics[width=0.7\textwidth]{Pictures/LSTcamerastructure.pdf}
  \caption{Different camera mechanical elements from \cite{2013LSTCamMech}.}
    \label{fig:LSTcammech}
\end{figure}

\subsection{MST}



\subsection{SST}
\section{Reconstruction and Analysis tools} \label{sec:ctaanalysis}
\subsection{Monte Carlo simulations}
\subsubsection{CORSIKA}
\subsubsection{sim\_telarray}
\subsection{Low level analysis: ctapipe}
\subsection{Science analysis: ctools}

\end{document}
