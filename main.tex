%%%%%%%%%%%%%%%%%%%%%%%%%%%%%%%%%%%%%%%%%
%  My documentation report
%  Objective: Explain what I did and how, in order to help someone continue with the investigation
%
% Important note:
% Chapter heading images should have a 2:1 width:height ratio,
% e.g. 920px width and 460px height.
%
% The images can be found anywhere, usually on sky surveys websites or the
% Astronomy Picture of the day archive http://apod.nasa.gov/apod/archivepix.html
%
% The original template (the Legrand Orange Book Template) can be found here --> http://www.latextemplates.com/template/the-legrand-orange-book
%
% Original author of the Legrand Orange Book Template:
% Mathias Legrand (legrand.mathias@gmail.com) with modifications by:
% Vel (vel@latextemplates.com)
%
% Original License:
% CC BY-NC-SA 3.0 (http://creativecommons.org/licenses/by-nc-sa/3.0/)
%%%%%%%%%%%%%%%%%%%%%%%%%%%%%%%%%%%%%%%%%
 
%----------------------------------------------------------------------------------------
%	PACKAGES AND OTHER DOCUMENT CONFIGURATIONS
%----------------------------------------------------------------------------------------
\documentclass[12pt,fleqn,aas_macros]{book} % Default font size and left-justified equations
\usepackage[top=3cm,bottom=3cm,left=3.2cm,right=3.2cm,headsep=10pt,letterpaper]{geometry} % Page margins

\usepackage{xcolor} % Required for specifying colors by name
\definecolor{ocre}{RGB}{52,177,201} % Define the orange color used for highlighting throughout the book

% Font Settings
\usepackage{avant} % Use the Avantgarde font for headings
%\usepackage{times} % Use the Times font for headings
%\usepackage{mathptmx} % Use the Adobe Times Roman as the default text font together with math symbols from the Sym­bol, Chancery and Com­puter Modern fonts

\usepackage{microtype} % Slightly tweak font spacing for aesthetics
\usepackage[utf8]{inputenc} % Required for including letters with accents
\usepackage[T1]{fontenc} % Use 8-bit encoding that has 256 glyphs

\usepackage{multirow}
\usepackage{hhline}

\usepackage[spanish,english]{babel}

%Golssary

\usepackage[acronym,nomain]{glossaries}
\makeglossaries

% Bibliography
\usepackage[backend=bibtex, style=nature, sorting=none]{biblatex}
\addbibresource{bibliography.bib} % BibTeX bibliography file
\defbibheading{bibempty}{}


% Other packages
\usepackage{subcaption}
\usepackage{graphicx}
\usepackage{subfiles}
\usepackage{amsmath}


\input{structure} % Insert the commands.tex file which contains the majority of the structure behind the template

\begin{document}

%Define Acronyms
\newacronym{vhe}{VHE}{Very High Energy}
\newacronym{pwn}{PWN}{Pulsar Wind Nebula}
\newacronym{pwne}{PWNe}{Pulsar Wind Nebulae}
\newacronym{cr}{CR}{Cosmic Ray}
\newacronym{ism}{ISM}{Interstellar Medium}
\newacronym{agn}{AGN}{Active Galaxy Nuclei}
\newacronym{grb}{GRB}{Gamma Ray Burst}
\newacronym{snr}{SNR}{Supernova Remnant}
\newacronym{sn}{SN}{Supernova}
\newacronym{sne}{SNe}{Supernovae}
\newacronym{dsa}{DSA}{Diffusive Shock Acceleration}
\newacronym{he}{HE}{High Energy}
\newacronym{uhecr}{UHECR}{Ultra High Energy Cosmic Ray}
\newacronym{uhe}{UHE}{Ultra High Energy}
\newacronym{ic}{IC}{Inverse Compton}
\newacronym{cmb}{CMB}{Cosmic Microwave Background}
\newacronym{ebl}{EBL}{Extragalactic Background Light}
\newacronym{eas}{EAS}{Extensive Air Showers}
\newacronym{dm}{DM}{Dark Matter}
\newacronym{sm}{SM}{Standard Model}
\newacronym{wmap}{WMAP}{Wilkinson Microwave Anisotropy Probe}
\newacronym{cdm}{CDM}{Cold Dark Matter}
\newacronym{wimp}{WIMP}{Weak Interacting Massive Particle}
\newacronym{susy}{SUSY}{Super Symmetry}
\newacronym{nfw}{NFW}{Navarro-Frenk-White}
\newacronym{comptel}{COMPTEL}{Imaging Compton Telescope}
\newacronym{egret}{EGRET}{Energetic Gamma Ray Experiment Telescope}
\newacronym{magic}{MAGIC}{Major Atmospheric Gamma Imaging Cherenkov Telescopes}
\newacronym{gc}{GC}{Galactic Center}
\newacronym{smbh}{SMBH}{Super Massive Black Hole}
\newacronym{bh}{BH}{Black Hole}
\newacronym{hess}{H.E.S.S.}{High Energy Stereoscopic System}
\newacronym{veritas}{VERITAS}{Very Energetic Radiation Imaging Telescope Array System}
\newacronym{msp}{MSP}{Millisecond Pulsar}
\newacronym{sfr}{SFR}{Star Formation Rate}
\newacronym{icm}{ICM}{Intracluster Medium}
\newacronym{lmc}{LMC}{Large Magellanic Cloud}
\newacronym{smc}{SMC}{Small Magellanic Cloud}
\newacronym{dsphe}{DSphe}{Dwarf Spheroidal Galaxies}
\newacronym{cgro}{CGRO}{Compton Gamma Ray Observatory}
\newacronym{integral}{INTEGRAL}{International Gamma-Ray Astrophysics Laboratory}
\newacronym{ibis}{IBIS}{Imager on-Board the INTEGRAL Satellite}
\newacronym{lat}{LAT}{Large Area Telescope}
\newacronym{gbm}{GBM}{Gamma-Ray Burst Monitor}
\newacronym{nai}{NaI}{Sodium Iodine}
\newacronym{bgo}{BGO}{Bismuth Germanate}
\newacronym{eas}{EAS}{Extensive Air Showers}
\newacronym{cta}{CTA}{Cherenkov Telescope Array}
\newacronym{airobicc}{AIROBICC}{Air shower Observation By angle Integrating Cherenkov Counters}
\newacronym{iact}{IACT}{Imaging Atmospheric Cherenkov Telescope}
\newacronym{hegra}{HEGRA}{High Energy Gamma Ray Astronomy}
\newacronym{amc}{AMC}{Active Mirror Control}
\newacronym{ccd}{CCD}{charged-coupled device}
\newacronym{fov}{FoV}{field of view}
\newacronym{pmt}{PMT}{Photomultiplier Tube}
\newacronym{dsr4}{DSR4}{Domino Ring Sampler version 4}
\newacronym{mhd}{MHD}{Magnetohydrodynamics}
\newacronym{hawc}{HAWC}{High-Altitude Water Cherenkov Observatory}
\newacronym{nsb}{NSB}{Night Sky Background}
\newacronym{lst}{LST}{Large Size Telescope}
\newacronym{mst}{MST}{Medium Size Telescope}
\newacronym{sst}{SST}{Small Size Telescope}
\newacronym{cfrp}{CFRP}{carbon fiber reinforced plastic}
\newacronym{scmst}{SC-MST}{Schwarzschild-Couder MST}
\newacronym{sc}{SC}{Schwarzschild-Couder}
\newacronym{dcmst}{DC-MST}{Davies-Cotton MST}
\newacronym{dc}{DC}{Davies-Cotton}
\newacronym{sipm}{SiPM}{Silicon Photomultiplier}
\newacronym{psf}{PSF}{Point Spread Function}
\newacronym{chec}{CHEC}{Compact High-Energy Camera}
\newacronym{fee}{FEE}{Front-end electronics}
\newacronym{mc}{MC}{Monte Carlo}
\newacronym{mva}{MVA}{Multi-variate analysis}
\newacronym{ml}{ML}{Machine Learning}
\newacronym{dl}{DL}{Deep Learning}
\newacronym{corsika}{CORSIKA}{\textbf{CO}smic \textbf{R}ay \textbf{SI}mulations for \textbf{KA}scade}
\newacronym{ts}{TS}{Test Statistics}
\newacronym{irf}{IRF}{Instrument Response Function}
\newacronym{rf}{RF}{Random Forest}
\newacronym{rms}{RMS}{root mean square}
\newacronym{mse}{MSE}{mean squared error}
\newacronym{em}{EM}{Expectation-Maximization}
\newacronym{roc}{ROC}{receiver operating characteristic}
\newacronym{mw}{MW}{Milky Way}
\newacronym{ulrigs}{ULRIGs}{Ultraluminous Infrarred Galaxies}
\newacronym{sfh}{SFH}{Stellar Formation History}
\newacronym{ms}{MS}{Magellanic System}
\newacronym{ksp}{KSP}{Key Science Project}
\newacronym{roi}{ROI}{Region of Interest}
\newacronym{cl}{CL}{Confidence Level}
\newacronym{asic}{ASIC}{Application-Specific Integrated Circuit}
\newacronym{dac}{DAC}{Digital to Analog Converter}
\newacronym{lvds}{LVDS}{Low-Voltage Differential Signalling}
\newacronym{scb}{SCB}{Slow Control Board}
\newacronym{adc}{ADC}{Analog to Digital Converter}


\title{Preparation of the exploitation of the Cherenkov Telescope Array: LST analysis and prospects for the studies of the Large Magellanic Cloud}


%----------------------------------------------------------------------------------------
%	TITLE PAGE
%----------------------------------------------------------------------------------------

\begingroup
\thispagestyle{empty}
\AddToShipoutPicture*{\put(0,0){\includegraphics[scale=1.25]{Pictures/LMCgrande.jpg}}} % Image background
\centering
\vspace*{5cm}
\par\normalfont\fontsize{32}{32}\sffamily\selectfont
\textbf{Preparation of the exploitation of the Cherenkov Telescope Array:}\\
{\LARGE LST analysis and prospects for the studies of the Large Magellanic Cloud}\par % Book title
\vspace*{0.7cm}
{\Huge María Isabel Bernardos Martín}\par % Author name
\endgroup

%----------------------------------------------------------------------------------------
%	COPYRIGHT PAGE
%----------------------------------------------------------------------------------------

\newpage
~\vfill
\thispagestyle{empty}

%\noindent Copyright \copyright\ 2020 María Isabel Bernardos Martín\\ % Copyright notice

\noindent \textsc{Doctoral Dissertation in Physics}\\

\noindent This research was done under the supervision of Dr. Carlos José Delgado with the financial support of the ... within a total duration of 4 years from April 1st 2016 to April 1st 2020.\\ % License information

\noindent \textit{First release, April 2020} % Printing/edition date


%Resumen

\newpage
\pagestyle{empty} % No headers
\chapter*{Resumen}
\subfile{Resumen}

%Abstract
\newpage
\pagestyle{empty} % No headers
\chapter*{Abstract}
\subfile{Abstract}




%----------------------------------------------------------------------------------------
%	TABLE OF CONTENTS
%----------------------------------------------------------------------------------------

\chapterimage{Pictures/LMCsuave.jpg} % Table of contents heading image

\pagestyle{empty} % No headers

\tableofcontents % Print the table of contents itself

%\cleardoublepage % Forces the first chapter to start on an odd page so it's on the right

\pagestyle{fancy} % Print headers again


%----------------------------------------------------------------------------------------
%	CHAPTER 1
%----------------------------------------------------------------------------------------

\chapterimage{Pictures/Crab_Nebula.jpg} % Chapter heading image

\chapter{Introduction to $\gamma$-Ray Astronomy} \label{cap:gammarayastro}
\subfile{Chapter1}

\chapterimage{Pictures/fermisurvey.jpg} % Chapter heading image

\chapter{Detection of $\gamma$-rays}
\subfile{Chapter2} \label{cap:detection}

\chapterimage{Pictures/cta.jpg} % Chapter heading image

\chapter{The Cherenkov Telescope Array} \label{cap:CTA}

\subfile{Chapter3}

\chapterimage{LST_02_crop.jpg} % Chapter heading image

\chapter[Analysis chain of Single Telescope data and sensitivity calculation for the LST1]{Analysis chain of Single Telescope data and sensitivity calculation for the LST1} \label{cap:LST1}
\subfile{Chapter4}

\chapterimage{Pictures/LMCnasa.jpg} % Chapter heading image

\chapter[Characterization of the Large Magellanic Cloud at TeV energies and prospects on Dark Matter detection for CTA]{Characterization of the Large Magellanic Cloud and prospects on Dark Matter detection for CTA}
\label{cap:LMC}
\subfile{Chapter5}

%Conclusiones

\newpage
\pagestyle{empty} % No headers
\chapter*{Conclusiones}
\subfile{Conclusiones}

%Conclusions
\newpage
\pagestyle{empty} % No headers
\chapter*{Conclusions}
\subfile{Conclusions}

\appendix
\chapterimage{LST_02_crop.jpg} % Chapter heading image
\chapter[The L1 Trigger System of the LST1]{The L1 Trigger System of the LST}
\label{app:trigger}
\subfile{AppendixA}

\chapter[Characterization of analogue ASIC for L1 trigger decision]{Characterization of analogue ASIC for L1 trigger decision}
\label{app:asic}
\subfile{AppendixB}

\chapter[Calibration method for the L1 analog trigger system of LST camera]{Calibration method for the L1 analog trigger system of LST camera}
\label{app:calib}
\subfile{AppendixC}

\chapter[Time calibration for implementation of the trigger distribution in the backplane network of LST camera]{Time calibration for implementation of the trigger distribution in the backplane network of LST camera}
\label{app:timecal}
\subfile{AppendixD}

\chapter[Random Forests]{Random Forests}
\label{app:rf}
\subfile{AppendixE}

\chapter[Evolution of LST1 camera uniformity]{Evolution of LST1 camera uniformity}
\label{app:lstcamevo}
\subfile{AppendixF}

\chapterimage{Pictures/LMCsuave.jpg} % Chapter heading image
\printbibliography
\printglossaries
\end{document}
