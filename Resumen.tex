\documentclass[main.tex]{subfiles}

\begin{document}
\glsresetall
\begin{otherlanguage}{spanish}
La Red de Telescopios Cherenkov (\gls{cta} en inglés) es un proyecto que tiene como objetivo la construcción de un observatorio para la observación de rayos gamma de muy alta energía. Estará compuesto de más de 100 telescopios de imagen de radiación Cherenkov atmosférica (\glspl{iact} en inglés), de tres tipos (tamaño pequeño, mediano y grande), repartidos en dos emplazamientos, en los hemisferios Norte y Sur, con el propósito de poder cubrir todo el cielo. \gls{cta} ha sido diseñado con el objetivo de abarcar un rango de energías desde los 20 GeV hasta los 300 TeV, con una resolución energética y espacial sin precedentes,  que le permitirán alcanzar una sensibilidad un orden de magnitud mayor que la actual generación de instalaciones que emplean la técnica de los \glspl{iact}.\\
Esta tesis está dedicada a dos frentes relativos a la preparación de \gls{cta}: La puesta en marcha del primer telescopio grande de \gls{cta}, el \gls{lst}1; y el estudio de la emisión de rayos $\gamma$ en la Gran Nube de Magallanes (\gls{lmc} en inglés) con el objetivo de hacer predicciones de los resultados que se obtendrán tras las observaciones con \gls{cta} de esta galaxia, tanto relativo a fuentes astrofísicas de rayos $\gamma$, como a una posible detección de señal de aniquilación de materia oscura.\\   
La primera parte de la tesis se centra en el primer prototipo de \gls{lst}, instalado en el emplazamiento Norte de \gls{cta} en la isla de La Palma, cuya primera luz tuvo lugar en Diciembre de 2018.
Durante el trabajo de esta tesis, he participado en varias tareas relativas a la puesta en marcha del telescopio. Principalmente, me he dedicado al desarrollo de las herramientas de software para el análisis de los datos del \gls{lst}, durante su tiempo de operación en solitario. La observación con un solo telescopio Cherenkov requiere de unas técnicas de reconstrucción de eventos específicas, por ello, ha sido necesario desarrollar todo un paquete de herramientas dedicado al \gls{lst}1, como complemento al software oficial de \gls{cta}. Este software ha sido desarrollado y verificado utilizando simulaciones de Monte Carlo especialmente producidas para el \gls{lst} y, posteriormente, ha podido ser aplicado con éxito a los datos reales recogidos con el \gls{lst}1 a lo largo de las tres campañas de toma de datos de la Nebulosa del Cangrejo que tuvieron lugar entre Noviembre de 2019 y Febrero de 2020. En el capítulo \ref{cap:LST1} se presento una descripción completa de la cadena de análisis, y se muestran sus resultados aplicada a simulaciones y a los datos reales del \gls{lst}1. Además, he desarrollado un método alternativo para el cálculo de los parámetros de Hillas de las imágenes de cascadas Cherenkov, que no requiere la aplicación previa de un método de limpieza (o `cleaning') de las imágenes, y permite recuperar información de cascadas con pocos fotones. Los resultados de este método comparados con el método tradicional, aplicado tanto a simulaciones como a datos reales son presentados también en este capítulo. Además, como trabajo complementario, he participado en algunas tareas relativas a la puesta en marcha de la cámara del \gls{lst}1, cuya estructura mecánica y parte de la electrónica han sido diseñadas y construidas en el CIEMAT. Los apéndices \ref{app:asic}, \ref{app:calib} y \ref{app:timecal} están dedicados a estas tareas, que consistieron en una caracterización de los \glspl{asic} del nivel L1 del sistema de trigger del telescopio, y a la calibración del de dicho sistema de trigger.\\

La segunda parte de esta tesis está dedicada a una caracterización de la emisión en rayos $\gamma$ de la \gls{lmc} a las energías de \gls{cta}. El estudio detallado de esta galaxia, con más de 300 horas de observación asignadas, es uno de los principales proyectos científicos de \gls{cta}. Para poder hacer una estimación de los resultados científicos que se obtendrán con este proyecto, ha sido necesario hacer una recopilación de la información obtenida por otros telescopios (tanto en rayos $\gamma$ como en otras longitues de onda) de la \gls{lmc} con el objetivo de construir un modelo de emisión extrapolado a las energías de \gls{cta}. El modelo de emisión construido contiene fuentes conocidas de rayos $\gamma$, como Nebulosas de Viento de púlsar (\gls{pwne}) o remanentes de supernova (\glspl{snr}) detectadas por otros telescopios (H.E.S.S. y Fermi-LAT), una emisión difusa de rayos $\gamma$ producida por la interacción de rayos cósmicos con el medio interestelar, y una población sintética de \gls{pwne} producida con el objetivo de estimar el número de nuevas fuentes de este tipo que podrán ser detectadas por \gls{cta}. Este modelo de emisión ha sido utilizado para realizar simulaciones de las observaciones con \gls{cta} de la región de la \gls{lmc}, que después se han ajustado al modelo utilizando un método de máxima verosimilitud para obtener estimaciones de la sensibilidad de \gls{cta} a la detección de las diferentes fuentes. El modelo de emisión y los resultados de este ajuste han sido utilizados posteriormente, como un fondo sobre el que estudiar de las posibilidades de \gls{cta} para detectar una señal de materia oscura producida por la aniquilación de partículas pesadas que interaccionan débilmente (\glspl{wimp}) en la \gls{lmc}. La descripción del modelo de emisión desarrollado y de las ténicas de simulación y ajuste, así como los resultados obtenidos están contenidos en el capítulo \ref{cap:LMC} de esta tesis. 
\\

A modo de introducción y puesta en contexto de la temática de la tesis, los tres primeros capítulos están dedicados al estado-del-arte de la astrofísica de rayos $\gamma$ (capítulo \ref{cap:gammarayastro}), las ténicas de detección y actual generación de telescopios de rayos $\gamma$ (capítulo \ref{cap:detection}), y a una descripción detallada de \gls{cta} (capítulo \ref{cap:CTA}). 
\end{otherlanguage}
\end{document}
