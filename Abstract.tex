\documentclass[main.tex]{subfiles}

\begin{document}
\glsresetall

The \gls{cta} is an ambitious project with the aim to build an observatory for the detection of very high energy $\gamma$-rays. It will be integrated by more than 100 \glspl{iact}, in three sizes (small, medium, large), situated in two locations, one for each hemisphere, with the purpose of covering the full sky. \gls{cta} has been designed to achieve an energy range from 20 GeV to 300 TeV, with energy and agular resolution without precedents, which will allow to reach sensitivities one order of magnitude better than the current generation of \gls{iact} facilities.\
This thesis is dedicated to two subjects relative to the preparation of \gls{cta} for operation: The commissioning of the first \gls{lst} of \gls{cta}; and the study of the very high energy emissión from the \gls{lmc} at the energies of \gls{cta}, in order to make predictions on the results that will be obtained from the survey of this galaxy, relative both to astrophysical $\gamma$-ray sources and to the detection of a possible signal from \gls{dm} annihilation.\\

The first part of the thesis is focused on the first prototype of the \gls{lst}, installed in the \gls{cta} North site, in La Palma island, which first light occured in December 2018.
During the time of this thesis, I have participated in several tasks relative to the commissioning of the \gls{lst}1, mainly related to the development of the software tools for the analysis of the telescope data during its single telescope operations. Observations with a single \gls{iact} require dedicated event reconstruction techniques, thus, a specific tool package has been developed for \gls{lst}1, as a complement to the offical \gls{cta} software. The analysis chain has been developed and verified using Monte Carlo simulations specially produced for the \gls{lst}1, and afterwards it has been succesfully applied to real data taken during the three observation campaigns of the Crab Nebula, which took place between November 2019 and February 2020. In chapter \ref{cap:LST1} I give a description of the analysis tools, together with the results when applied to simulations and to real data from the Crab campaigns. In addition, I have developed an alternative method for the calculation of the Hillas parameters of Cherenkov showers, which do not require a cleaning method of the shower images, allowing to recover information from showers with low number of photons. The results of this method, compared to the traditional method which performs a tailcuts cleaning, applied both to simulations and to real data are also presented in this chapter. As a complementary work, I have participated in some activities related to the commissioning of the \gls{lst}1 camera, which mechanical structure and some parts of the electronics were designed and produced at CIEMAT. Appendices \ref{app:asic}, \ref{app:calib} and \ref{app:timecal} are dedicated to these tasks, consisting on a characterization of the \glspl{asic} for the L1 trigger system, and the calibration of such trigger system.\\

The second part of the thesis is dedicated to a characterization of the $\gamma$-ray emission of the \gls{lmc} at the energies of \gls{cta}. A deep survey of this galaxy, with more than 300 hours of assigned observation time, is one of the \glspl{ksp} of \gls{cta}. To perform an estimation of the future scientific products which will be provided by this survey, it has been necessary to collect the results obtained by other telescopes ($\gamma$-rays and other wavelenghts) of the \gls{lmc}, to build an emission model extrapolated to the energies of \gls{cta}. The emission model produced is composed by known $\gamma$-ray sources, such as \gls{pwne} and \glspl{snr} detected by other telescopes (H.E.S.S. and Fermi-LAT), a diffuse $\gamma$-ray emission produced by \glspl{cr} interacting with the \gls{ism}, and a synthetic population of \gls{pwne} produced with the purpose of estimate the number of new of such sources that will be detected by \gls{cta}. This emission model has been used to simulate observations of the region of interest of the \gls{lmc} with \gls{cta}, which afterwards has been fitted to the model with a maximum likelihood method to obtain estimations on the sensitivity of \gls{cta} to the detection of the different sources. The emission model and the results of this fitting have been used then as a background for the study of the possibilities of \gls{cta} to detect a \gls{dm} signal produced by the annihilation of \glspl{wimp}. The description of the emission model developed, the analysis and simulation techniques, and the results obtained are presented in chapter \ref{cap:LMC}.

As an introduction and to provide context for this thesis, the first three chapters are dedicated to the State-of-the-Art of $\gamma$-ray Astrophysics (chapter \ref{cap:gammarayastro}), detection techniques and current generation of $\gamma$-ray instruments (chapter \ref{cap:detection}) and a detailed description of \gls{cta} (chapter \ref{cap:CTA}).

\end{document}
