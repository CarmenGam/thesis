\documentclass[main.tex]{subfiles}
\begin{document}
\glsresetall

\section{Introduction}

The first \gls{lst} was inaugurated in October 2018 and since then has been in commissioning phase. Being the first \gls{cta} telescope installed on site (in La Palma island), and it is expected to be operating on its own until LST 2-4 are built. This mean that LST1 will need its own analysis chain in \textit{mono} mode, which differs in several aspecst from the stereo analysis included in the benchmark analysis tools of \gls{cta} (see section \ref{sec:ctapipe}). Although single telescope observations present a big challenge, specially regarding source position reconstruction and $\gamma$-hadron separation, it is expected that LST1 performance, thanks to its size and camera design, will be competitive enoughh to offer scientific results in the time it will be operating alone.\\
This chapter will present a considerable amount of the work done during this thesis, which includes the development of the code for the single telescope analysis for LST1, the calculation of LST1 sensitivity based on \gls{mc} simulations, the development of a new technique for Hillas Parameters calculation without cleaning using the Expectation-Maximization algorithm and the application of the analysis chain to real LST1 data. 

\section{The LST1 analysis chain overview}

The analysis software for the single telescope analysis of LST1 named \textit{cta-lstchain} has been developed before the necessity of specific tools for single telescope analysis not included in \textit{ctapipe}, the benchmark analysis tools for low level data of \gls{cta}. It is written as a python package which heavily relies on \textit{ctapipe}, and it is structured in several modules containing functions destined to the different parts of the analysis. The first version of \textit{cta-lstchain} was written by the author of this thesis and many contributors have joined the project over the last two years to improve and optimize the repository to its current version, which is able to perform every analysis step both for simulated \gls{mc} data and real data. The analysis chain is divided in several steps, each of which can be executed through a python script which requires certain inputs and calls for the appropriate functions. All the configuration parameters of the different elements of the analysis are given through configuration files, which can be edited by the user or else a standard configuration will be used. The input of the analysis chain are the raw data files of LST1 events, which for \gls{mc} data are \textit{sim$\_$telarray} files and for real data are \textit{zfits} files with a similar internal format. The files contain the full information available per pixel, known as \textit{waveform} and which is the digitized signal amplitude vs. time sample for every triggered event, together with \gls{mc} information in the case of a simulated file (such as the true energy, source position, number of simulated events, and so on), or recorded information from the different telescope subsystems in the case of real data (such as time, pointing, trigger type...). Throughout the analysis chain, the data including the images and image parameters, is stored in containers designed in \textit{cta-lstchain} specifically for LST1, which can be dumped into \textit{pandas} dataframes and saved in \textit{hdf5} files.\\
The main steps of the analysis chain can be summarized as: \\

\begin{itemize}
\item\textbf{Calibration:} The waveforms of each pixel in the camera should be integrated after pedestal subtraction, and converted to number of photoelectrons. Also, the timing information of the signal is obtained. 
\item\textbf{Image cleaning and parametrization:} The images in the camera contain pixels with light not related to the cherenkov event, so a cleaning must be applied to remove them. Afterwards, the photon distribution in the image is used to calculate the Hillas parameters.
\item\textbf{Energy and direction reconstruction:} The energy and direction reconstruction of the triggered events are performed using a multidimensional regresssion technique based on \glspl{rf}. A set of simulated diffuse $\gamma$ events are used to train the \gls{rf}. 
\item\textbf{$\gamma$-hadron separation:} For the $\gamma$-hadron separation, a multidimensional \gls{rf} classifier is used. Sets of simulated $gamma$ and proton events, which energies and directions have been reconstructed in the previous step, are used to train the classifier.
\end{itemize}

\subsection{Calibration}

In the calibration phase, the raw signal known as \textit{waveform} is integrated after background subtraction, to obtain a total number of counts per pixel, which afterwards is multiplied by a factor to be converted in number of photoelectrons. Depending on if \gls{mc} data or real data is being analyzed, the following steps vary and will be explained for each case.

\subsubsection{Signal extraction}

For every triggered event, the signal in each pixel is recorded in a 40 samples window from the 4096 samples \gls{dsr4}. This signal contains information not only from the Cherenkov light, but also from background light from \gls{nsb} and from the intrinsic noise induced by the readout chain. Before integrating the signal, it is necessary to subtract this \textit{pedestal}. For simulated \gls{mc} events the pedestal value for each pixel in the camera is already stored, but for real data it is necessary to take special pedestal runs, with randomly activated trigger. Pedestal events are used to calculate the mean pedestal value for each \gls{dsr4} sample so typically, around 1000 events are needed to fill the ring. In \textit{cta-lstchain} a specific script is used to extract the pedestal values from pedestal runs and they are stored in a \textit{hdf5} file to be used later in the calibration.\\
Once the pedestal is subtracted from the signal, the signal peak can be integrated. Typically a smaller window of a few samples around the maximum is used for the integration, which can be performed with one of the several integrators implemented in \textit{ctapipe}. By default, the integrator used in \textit{cta-lstchain} is the \textit{NeighborPeakWindowSum}, which sums the signal in a window around the peak defined by the waveform in neighbouring pixels. This allow to avoid integrating peaks which can arise from fluctuations. The default width of the integration window for this integrator is of 7 samples. When analyzing real data, the first and last two samples of the waveform should be dropped, to avoid integrating the spikes that can arise from ?? ¿¿. These calibration steps are performed for the two gain channels of \gls{lst}1 camera, if available. By default, the signal used is the high gain, unless it is saturated, meaning the maximum is above 4096 counts. In this case, the channel is switched to low gain for that pixel. 

\subsubsection{Conversion to photoelectrons}

Once the signal amplitude is extracted, it must be converted from DC counts to photoelectrons through a calibration factor, which is different for each pixel and channel. For simulated \gls{mc} data, these factors are stored in the simulated file and the conversion can be done simply by multiplying the image in DC counts by the factor of each pixel. For real data, special calibration runs must be taken to calculate these factors. Calibration events are taken using Ultraviolet light pulses fired from the CaliBox \cite{2015CaliBox}, \cite{2019CaliBox} located in the mirror dish. The calculation of the calibration coefficients is made using the F-factor method \cite{1997calibrationPMT}, which assumes that the distribution of photoelectrons in a \gls{pmt} follows a poissonian statistics with a mean $N$ and a \gls{rms} of $\sqrt{N}$. The signal distribution, however, will be deviated from a poissonian statistics with a wider \gls{rms} due to a excess-noise factor F, which is different for each \gls{pmt} and whould be measured in the laboratory. The relation between the relative widths of the two distributions can be written as:

\begin{equation}
  F \cdot \frac{1}{\sqrt{N}} = \frac{\sigma_{Q}}{<Q>}
  \label{eq:ffactor}
\end{equation} 

Where $<Q>$ is the mean value of the pixel signal and $\sigma_{Q}$ its \gls{rms}. The calibration coefficients will come from the relation between the number of photoelectrons $N$ and the mean pixel signal $<Q>$, which from \ref{eq:ffactor}, and taking into account that the signal must be corrected from pedestal:

\begin{equation}
  \frac{N}{<Q>} = F^{2}\frac{<Q> - <ped>}{\sigma_{Q}^{2} - \sigma_{ped}^{2}}
\end{equation}

The values of $<Q>$, $\sigma_{Q}$, $<ped>$ and $\sigma_{ped}$ are calaculated from a sufficiently large number ($\sim$ 1000) of calibration and pedestal events, using a specific script from \textit{cta-lstchain}, which stores the resulting calibration coefficients in an HDF5 file to be used later in the calibration of data runs. For the LST1, the value of F factor used is the mean value for all \gls{pmt} which is 1.2.

\subsection{Image cleaning and parametrization}

\subsection{Reconstruction of energy and source position}

\subsection{$\gamma$-hadron separation}

\section[Expectation-Maximization method for Hillas Parameters...]{Expectation-Maximization method for Hillas Parameters calculation without cleaning}
\section{Sensitivity calculation}
\section{Results on real data}

\end{document}
