\documentclass[main.tex]{subfiles}

\begin{document}
\glsresetall
\begin{otherlanguage}{spanish}

A lo largo del desarrollo de esta tesis, he contribuído a la puesta en marcha de la Red de Telescopios Cherenkov (\gls{cta} en inglés), aportando mi granito de arena a la astrofíscia de rayos $\gamma$ de muy alta energía.\\
El prototipo del telescopio grande de \gls{cta}, el \gls{lst}1, se encuentra en fase de comisionado desde Diciembre de 2018, y desde sus inicicios, me he involucrado en diferentes tareas para su puesta en marcha. La mitad del trabajo que he desarrollado durante mi tesis, ha estado destinado al desarrollo y prueba del software de análisis de datos de este telescopio, durante su operación en solitario. Este software, actualmente en una versión muy avanzada, permite tratar los datos recogidos por el telescopio, desde el más bajo nivel (R0), hasta la reconstrucción de la información de la partícula primaria (es decir, su energía, dirección y distinción entre $\gamma$s y hadrones, nivel DL2). La reconstrucción de los datos se realiza en varias fases. En primer lugar, los datos recogidos en cada pixel de la cámara son calibrados, de manera que es posible realizar la conversión entre cuentas del \gls{adc} y fotoelectrones. Una vez obtenida la imagen de la cascada Cherenkov en la cámara, esta es parametrizada según el método de Hillas. Estos parámetros, junto a la información del tiempo de llegada de la luz a cada pixel, son utilizados para reconstruir la energía, dirección y tipo de partícula primaria con la ayuda de técnicas de Machine Learning, los \glspl{rf}. Mi contribución a este trabajo, ha consistido en la escritura y verifación de gran parte del código del software, especialmente en sus primeras fases de desarrollo. He utilizado la cadena de análisis, aplicada a simulaciones de Monte Carlo, para derivar el rendimiento del telescopio, que alcanza una resolución en la energía del 20\% en el rango desde $\sim 100$ GeV hasta $\sim 1$ TeV. Con respecto a la resolución angular, esta es de hasta $\sim 0.2º$ en el mismo rango de energías. También he derivado la sensitividad del telescopio a la detección de una fuente puntual tras 50h de observación, que se acerca al 10\% del flujo de la Nebulosa del Cangrejo en el rango entre 100 GeV y 1 TeV. He utilizado la misma cadena de análisis para reconstruir los datos obtenidos durante las tres campañas de toma de datos llevadas acabo con el \gls{lst}1, en las que se ha alcanzado una significancia de detección de la Nebulosa del Cangrejo de entre 6-10 $\sigma/\sqrt{h}$.\\
Adicionalmente, he desarrollado un método alternativo para el cálculo de los parámetros de Hillas basado en un algoritmo de Esperanza-Maximización (\gls{em} en inglés), que no requiere la aplicación de cortes en el número de fotoelectrones de los pixeles para eliminar el fondo de la noche. Aplicando este método, he demostrado que se puede obtener un rendimiento similar al obtenido con el modo tradicional, evitando ajustar los parámetros del corte en los fotoelectrones de las imágenes, que claramente afectan al rendimiento de la reconstrucción. Además, se ha visto que el método de \gls{em}, permite realizar una mejor reconstrucción de cascadas con pocos foto electrones, al favorecer la clasificación como $\gamma$s de un mayor número de eventos. Al hacer el análisis con \gls{em}, se pueden recuperar hasta un 40\% de eventos de $\gamma$s que tras un corte de gammaness > 0.7, serían descartados por el método tradicional. Por otro lado, no obstante, tiene la desventaja de asignar un valor de gammaness más alto a algunos eventos de protones. Para el mismo corte de gammaness > 0.7, habría un 10\% más de protones que serían erróneamente clasificados como $\gamma$s.\\
Aplicado a datos reales del \gls{lst}1, el método de \gls{em} muestra que puede igualar en sensibilidad al método tradicional, sin hacer cortes en la intensidad de las cascadas, y recuperando por tanto un gran número de eventos, ya que típicamente, para obtener un buen rendimiento con el método tradicional, es necesario descartar cascadas con menos de 500 fotoelectrones. Como consecuencia, presumiblemente este método permitiría conservar un mayor número de eventos de baja energía, reduciendo el umbral de energía del telescopio.\\

Como segunda parte de la tesis, realizada en paralelo con la anterior, he participado en el proyecto científico de la Gran Nube de Magallanes (\gls{lmc} en inglés) de \gls{cta}. Este proyecto ha asignado más de 300 horas de obervación a esta galaxia, con las que se pretende alncanzar tres principales objetivos científicos: El estudio de las potentes fuentes de rayos $\gamma$ presentes en la \gls{lmc} y descubrimiento de otras nuevas, el estudio de la inyección y propagación de rayos cósmicos en la \gls{lmc} a través de la detección de una emisión difusa; y la posible detección de una señal de aniquilación de Materia Oscura en la galaxia. Para garantizar el éxito de este proyecto, ha sido necesario crear un modelo de la emisión esperada en la \gls{lmc} en el rango de energías de \gls{cta}. El modelo de emisión construído comprende fuentes puntuales conocidas, emisión difusa producida por rayos cósmicos interaccionando con el medio interestelar y una población sintética de nebulosas de viento de púlsar (\gls{pwne}). El modelo ha sido utilizado para realizar una simulación de observación con el software de \gls{cta}, \textit{ctools}, que reproduce el aspecto que tendrán los datos obtenidos al completar el tiempo de observación asignado al proyecto. Con esta simulación como `datos', se ha realizado un ajuste al modelo de emisión, utilizando el método de máxima verosimilitud, para obtener predicciones en las posibilidades de \gls{cta} para detectar y estudiar las diferentes fuentes de rayos $\gamma$ en la \gls{lmc}. Los resultados muestran que las potentes fuentes puntuales conocidas serán detectadas con una gran significancia, permitiendo estudiar sus características espectrales en el rango de energías desde 100 GeV hasta 100 TeV. También, se estima que al menos una decena de nuevas fuentes del tipo de \gls{pwne} serán detectadas. Con respecto a la emisión difusa de rayos cósmicos, el modelo utilizado para recrear la componente leptónica producida por scattering Compton inverso de rayos cósmicos cargados parece demasiado débil como para poder ser distinguida del fondo instrumental en el análisis. La componente hadrónica, producida por decaimiento de piones, ofrece resultados más optimistas, pudiendo ser detectada con alta significancia, especialmente en las regiones donde hay más acumulación de gas en la galaxia.\\
Además del estudio de las fuentes astrofísicas, se ha utilizado el modelo de emisión creado como un fondo sobre el que estimar si \gls{cta} podría ser capaz de detectar una señal de rayos $\gamma$ producida por la aniquilación de materia oscura formada por partículas pesadas que interaccionan débilmente (\glspl{wimp}). Para ello, se han construído diferentes modelos de perfiles de materia oscura en la \gls{lmc}, combinados con diferentes canales de aniquilación, comunmente utilizados en la literatura, para estudiar el espacio de parámetros entre sección eficaz de aniquilación y masa de la partícula de Materia Oscura que podría ser estudiado por \gls{cta}. Los resultados de este análisis muestran que, salvo en los modelos en los que la materia oscura está muy concentrada en un volumen muy pequeño (modelos, por otro lado, poco realistas), \gls{cta} no sería capaz de descartar el modelo típico de \glspl{wimp} con la sección eficaz canónica. \\
Este trabajo servirá como base para el proyecto de la \gls{lmc} dentro de \gls{cta}, y abre el camino para la realización de análisis más realistas y que se acercarńa más a los resultados con datos reales. Los próximos pasos serían hacer un análisis `ciego' de las fuentes puntuales, estudiar en detalle las características que se podrán derivar de las fuentes conocidas, incluir en el model el remanente de supernova SNR1987A, incluir una población desconocida de remanentes de supernova, etcétera.
Los resultados de este trabajo han dado como fruto un artículo dentro del consorcio \gls{cta} que se encuentra actualmente en fase de preparación, y del cual soy una de las principales autoras.
\end{otherlanguage}
\end{document}
